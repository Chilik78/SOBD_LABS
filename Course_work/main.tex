\documentclass[a4paper,14pt,oneside,openany]{memoir}

% ===== Настройки и вспомогательные пакеты =====
\usepackage{coursework}  % включает fontspec, polyglossia, biblatex и шрифты
\usepackage{csquotes}    % для корректной работы biblatex
\usepackage{graphicx}
\usepackage{amsmath}
\usepackage{array}
\usepackage{longtable}
\usepackage{float}
\usepackage{indentfirst}
\usepackage{setspace}
\usepackage{caption}
\usepackage{tocloft}
\usepackage{ragged2e}

\begin{document}

% ========== Титульный лист ==========
% Информация для титульного листа и листа задания
% Следует заполнить актуальными данными

% Кафедра, на которой читается дисциплина
\setdepartment{Электронно-вычислительные машины и системы}

% Инициалы и фамилия заведующего кафедрой
\setdepartmentchairname{А.Е.~Андреев}

% Полное название дисциплины
\setsubject{Системы обработки больших данных}

% Название темы курсовой работы
\settheme{Исследование датасета Big Sales Data с использованием фреймворка Apache Spark}

% Фамилия Имя Отчество студента
\setstudentname{Челядинов Дмитрий Владимирович}

% Инициалы и фамилия студента
\setstudentinitials{Д.В.~Челядинов}

% Учебная группа студента
\setgroupname{САПР-1.1}

% Инициалы и фамилия преподавателя, проверяющего курсовую работу
\setadvisorname{П.Д.~Кравченя}

\maketitlepage

% ========== Оглавление ==========
\renewcommand{\contentsname}{Содержание}
\tableofcontents*
\addtocontents{toc}{\vspace{1\baselineskip}}
\clearpage

\justifying

% ========== Введение ==========
\chapter*{Введение}
\addcontentsline{toc}{chapter}{\MakeUppercase{Введение}}
% Содержание введения
%
\vspace{\baselineskip}

Актуальность данной курсовой работы обусловлена возрастающей потребностью в разработке эффективных методов машинногообучениядля обработки больших данных \cite{karau2015spark, damji2020learning, chambers2018spark} и извлечения полезных знаний из пользовательского контента \cite{koirala2020pyspark}. Особую значимость приобретают комплексные подходы, сочетающие предварительную обработку данных и построение прогностических моделей на распределенных вычислительных платформах.

В связи с этим целью данной курсовой работы является исследование и реализация полного цикла машинного обучения на больших данных о поведении пользователя на рынке электронной коммерции с использованием фреймворка Apache Spark \cite{spark2022official, armbrust2015spark}.

Для достижения поставленной цели выдвинуты следующие задачи:
\begin{enumerate}
    \item Загрузка и первичное исследование структуры данных из распределенной файловой системы HDFS;
    \item Выполнение базовых преобразований и очистки данных для подготовки к машинному обучению;
    \item Применение алгоритмов машинного обучения на больших данных;
    \item Анализ и оценка качества построенных прогностических моделей;
    \item Визуализация результатов и подготовка выводов по исследованию.
\end{enumerate}

Впервом разделе рассмотрена более подробно постановка задачи и проведен обзор современных методов машинного обучения на больших данных. Во втором разделе описана методика предварительной обработки данных и выполнена верификация качества очистки. В третьем разделе представлены результаты применения алгоритмов машинного обучения, а также анализ эффективности построенных моделей. В заключении работы сформулированы общие выводы по результатам проведенного исследования.

% ========== Основные главы ==========

% Содержание первой главы

\chapter{\MakeUppercase{Разведочный анализ данных с использованием PySpark}}\label{ch:first}

\section{Постановка задачи разведочного анализа}\vspace{\baselineskip}

Задачей данной главы является проведение разведочного анализа большого набора данных о поведении пользователей электронной коммерции с использованием возможностей фреймворка Apache Spark.
Основные задачи разведочного анализа заключаются в следующем:

\begin{itemize}
\item загрузка данных из распределённой файловой системы HDFS и формирование единого датафрейма;
\item исследование структуры, схемы и качества данных;
\item выявление пропусков, дубликатов и некорректных значений;
\item преобразование типов данных и создание производных признаков;
\item предварительная оценка распределений количественных признаков и анализа категориальных данных;
\item подготовка очищенного и структурированного набора данных для последующего применения алгоритмов машинного обучения.
\end{itemize}

Результатом данного этапа является построение целостного представления о данных и формирование корректной основы для дальнейших шагов анализа и моделирования.
\vspace{\baselineskip}

\section{Описание исходного датасета}\vspace{\baselineskip}

В работе используется датасет <<eCommerce behavior data from multi category store>>, доступный на платформе Kaggle \cite{kaggle2023dataset}. Набор данных состоит из одного датасета — 2019-Nov.csv. В дальнейшем его название изменится на dataset.csv.

Датафрейм включает следующие ключевые признаки (таблица \ref{tab:features}):

\begin{table}[H]
    \centering
    \begin{tabularx}{\textwidth}{|X|X|}
        \hline
        Признак & Описание \\
        \hline
        event\_time & Время, когда произошло событие (UTC). \\
        \hline
        event\_type & Вид события. \\
        \hline
        product\_id & Идентификатор продукта. \\
        \hline
        category\_id & Идентификатор категории продукта. \\
        \hline
        category\_code & Таксономия категории товара (кодовое название). \\
        \hline
        brand & Строка с названием бренда. \\
        \hline
        price & Цена продукта. \\
        \hline
        user\_id & Идентификатор пользователя. \\
        \hline
    \end{tabularx}
    \caption{Описание признаков датасета}
    \label{tab:features}
\end{table}

Совокупный объём данных составляет более 67 миллионов строк. Загруженные данные изначально имеют строковые типы и разнородные форматы идентификаторов. Чтение и демонстрация исходных данных производилось с помощью следующего кода:

\begin{code}
path = "hdfs://namenode:9000/user/dchel/dataset.csv"
df = (spark.read.format("csv")
      .option("header", "true")
      .load(path)
)
df.show()
\end{code}

Данный фрагмент показывает, что:

\begin{itemize}
\item исходные данные находятся в HDFS;
\item происходит чтение csv файла dataset.csv;
\item происходит вывод первых 20 строк датафрейма в консоль.
\end{itemize}

Полный вывод результата выполнения код представлен на рисунке \ref{fig:ExampleData1}.

\begin{figure}[H]
    \centering
    \includegraphics[width=0.8\textwidth]{Content/Images/Analyze/ExampleData.png}
    \caption{Данные из датасета}
    \label{fig:ExampleData1}
\end{figure}

В ходе анализа полей датасета, с помощью команды \texttt{df.select(
"event\_type", "product\_id", "category\_id", "category\_code
", "brand", "price")} были выбраны следующие поля:  \texttt{event\_type},  \texttt{product\_id},  \texttt{category\_id},  \texttt{category\_code},  \texttt{brand}, \texttt{price} (см. рис. \ref{fig:ExampleDataSelect}).

\begin{figure}[H]
    \centering
    \includegraphics[width=0.7\textwidth]{Content/Images/Analyze/ExampleDataAfterSelect.png}
    \caption{Данные из датасета после select}
    \label{fig:ExampleDataSelect}
\end{figure}

Структура данных была изучена с использованием команды \texttt{df.printSc
hema()}, что позволило определить типы полей и выявить их потенциальную неоднородность. Так, поля \texttt{event\_type}, \texttt{product\_id}, \texttt{category\_id}, \texttt{category\_code}, \texttt{brand}, \texttt{price} и текстовые поля загружаются как строки, что указывает на возможное наличие разнородных форматов данных. Структура представлена на рисунке (см. рис. \ref{fig:DataFrameScheme}).

\begin{figure}[htbp]
    \centering
    \includegraphics{Content/Images/Analyze/DataFrameScheme.png}
    \caption{Структура таблицы после загрузки данных}
    \label{fig:DataFrameScheme}
\end{figure}

Более детальное изучение содержимого выполнялось уже на этапе разведочного анализа при помощи выборочного просмотра записей df.show() (см. рис. \ref{fig:ExampleDataAfterShow}), анализа уникальных значений и регулярных выражений.

\begin{figure}[H]
    \centering
    \includegraphics[width=0.8\textwidth]{Content/Images/Analyze/ExampleDataAfterShow.png}
    \caption{Выборочный просмотр записей}
    \label{fig:ExampleDataAfterShow}
\end{figure}

Эти методы позволили установить, что:

\begin{itemize}
\item числовые идентификаторы различной длины (от 6 до 8 цифр);
\item только одно значение view во всех строках выборки;
\item множество значений NULL;
\item значительный разброс цен (от бюджетных товаров до премиальных);
\item числовые значения с двумя десятичными знаками.
\end{itemize}

Параметры Spark-сессии были настроены с учётом объёма данных \cite{karau2015spark, damji2020learning}: увеличены объёмы памяти драйвера и исполнителей. Это обеспечивает стабильную работу при чтении и трансформации больших датафреймов. На рисунке \ref{fig:SparkSession} показана конфигурация SparkSession.

В конфиге, указанном ниже, последние строчки указывают на то, что используется spark:

\begin{code}
def create_spark_configuration()-> SparkConf:
    user_name = "dchel"
    conf = SparkConf()
    conf.setAppName("Lab 1")
    conf.setMaster("local[*]")
    conf.set("spark.submit.deployMode", "client")
    conf.set("spark.executor.memory", "12g")
    conf.set("spark.executor.cores", "8")
    conf.set("spark.executor.instances", "2")
    conf.set("spark.driver.memory", "4g")
    conf.set("spark.driver.cores", "2")
    conf.set("spark.sql.catalog.spark_catalog.type",
    "hadoop")
    conf.set("spark.sql.catalog.spark_catalog.warehouse",
    f"hdfs:///user/{user_name}")
    conf.set("spark.sql.catalog.spark_catalog.io-impl",
    "org.apache.iceberg.hadoop.HadoopFileIO")
    return conf
\end{code}

\begin{figure}[H]
    \centering
    \includegraphics[width=.6\textwidth]{Content/Images/Analyze/SparkSession.png}
    \caption{Демонстрация работы сессии Spark}
    \label{fig:SparkSession}
\end{figure}

Также для работы с датасетом, он был предварительно загружен в hdfs. Обзор директории представлен на рисунке \ref{fig:DatasetInHadoop}.

\begin{figure}[H]
    \centering
    \includegraphics[width=\textwidth]{Content/Images/Analyze/DatasetInHadoop.png}
    \caption{Директория с данными для работы}
    \label{fig:DatasetInHadoop}
\end{figure}

\vspace{\baselineskip}

\section{Определение пропущенных значений и преобразование данных}\vspace{\baselineskip}

Анализ полноты данных показал наличие пропусков в ряде столбцов, преимущественно в числовых полях. Для оценки количества пропусков была использована служебная функция, выполняющая подсчёт NULL-значений:

\begin{code}
def count_nulls(data: DataFrame,
                column_name: str)-> None:
    null_counts = data.select(
    sum(col(column_name).isNull().cast("int"))
    ).collect()[0][0]

    not_null_counts = data.select(
        sum(col(column_name).isNotNull().cast("int"))
    ).collect()[0][0]

    print(f"Число колонок с NULL: {null_counts} "
          f"({100 * null_counts / (null_counts +
          not_null_counts):.2f}%)")
\end{code}

\vspace{\baselineskip}
Были применены следующие стратегии:

\begin{itemize}
\item удалены строки содержащие \texttt{NULL}-значения в столбцах \texttt{category\_co
de} и \texttt{brand} командой \texttt{data.dropna(subset=["category\_code
", "brand"])};
\item текстовый столбец \texttt{category\_id} был удален с помощью команды \texttt{df.drop("category\_id")}.
\end{itemize}

После очистки структура данных была расширена и дополнена новыми признаками. В частности, выполнено преобразование типов:

\begin{itemize}
\item численный перевод идентификаторов продуктов (\texttt{product\_id});
\item перевод кодов категории в массив строк (\texttt{category\_code});
\item численный перевод цены (\texttt{price}).
\end{itemize}

Дополнительно созданы следующие производные признаки:

\begin{itemize}
\item массив кодов категорий, полученный путём разбиения строки с использованием \texttt{split};
\item признак содержания вида продукта \texttt{contains\_appliances}, \texttt{contains\_computers}, \texttt{contains\_electronics}, \texttt{contains\_ki
tchen}, \texttt{contains\_smartphone};
\item булевый признак дороговизны продукта \texttt{is\_expensive};
\item булевый признак бюджетного продукта \texttt{is\_budget};
\item булевый признак среднебюджетного продукта \texttt{is\_mid\_range};
\item кол-во категорий, которые охватывают продукт \texttt{category\_count};
\item булевый признак просмотра продукта \texttt{is\_view};
\item булевый признак добавление продукта в корзину \texttt{is\_cart};
\item булевый признак покупки продукта \texttt{is\_purchase};
\item класс продукта \texttt{price\_range};
\item класс продукта в численном формате \texttt{price\_range\_numeric}.
\end{itemize}

Бинарные признаки создаются с помощью следующего скрипта:

\begin{code}
df = df.withColumn("is_expensive", when(col("price") > 200, 1)
    .otherwise(0))
df = df.withColumn("is_budget", when(col("price") < 50, 1)
    .otherwise(0))
df = df.withColumn("is_mid_range", when((col("price") >= 50) 
    & (col("price") <= 200), 1).otherwise(0))
df = df.withColumn("is_purchase", 
    when(col("event_type") == "purchase", 1).otherwise(0))
df = df.withColumn("is_view", 
    when(col("event_type") == "view", 1).otherwise(0))
df = df.withColumn("is_cart", 
    when(col("event_type") == "cart", 1).otherwise(0))
\end{code}

Количественные признаки создаются с помощью следующего скрипта:

\begin{code}
df = df.withColumn("category_count", 
                   col("contains_appliances").cast("int") + 
                   col("contains_computers").cast("int") + 
                   col("contains_electronics").cast("int") + 
                   col("contains_kitchen").cast("int") + 
                   col("contains_smartphone").cast("int"))
\end{code}

Категориальные признаки создаются с помощью следующего скрипта:

\begin{code}
df = df.withColumn("price_range", 
                   when(col("price") < 50, "budget")
                   .when(col("price") < 150, "affordable")
                   .when(col("price") < 300, "premium")
                   .otherwise("luxury"))
df = df.withColumn("price_range_numeric",
                   when(col("price_range") == "budget", 1)
                   .when(col("price_range") == "affordable",
                     2)
                   .when(col("price_range") == "premium", 3)
                   .otherwise(4))
\end{code}

Результаты продемонстрированы на рисунках \ref{fig:SchemeAfterProcessing1} и \ref{fig:ExampleDataAfterProcessing}.

\begin{figure}[H]
    \centering
    \includegraphics[width=0.5\textwidth]{Content/Images/Analyze/SchemeAfterProcessing.png}
    \caption{Структура таблицы после обработки данных}
    \label{fig:SchemeAfterProcessing1}
\end{figure}

\begin{figure}[H]
    \centering
    \includegraphics[width=1\textwidth]{Content/Images/Analyze/ExampleDataAfterProcessing.png}
    \caption{Фрагмент данных после обработки}
    \label{fig:ExampleDataAfterProcessing}
\end{figure}

Все преобразования были объединены в функцию, позволяющую повторно применять трансформации к датафрейму. Скрипт вынесен в Приложение \ref{app:transform_data}.


\vspace{\baselineskip}

\section{Анализ распределений, выбросов и категориальных признаков}\vspace{\baselineskip}

Для количественного признака \texttt{price}, была построена гистограмма с использованием Spark и последующей визуализацией в библиотеке Matplotlib (см. Приложение \ref{app:hist}).

Фрагмент кода расчета данных для построения гистограммы распределения:

\begin{code}
min_value = data.selectExpr(f"min({column})").collect()[0][0]
max_value = data.selectExpr(f"max({column})").collect()[0][0]
bin_size = (max_value - min_value) / num_bins

data_with_bin = data.selectExpr("*", 
    f"floor(({column} - {min_value}) / {bin_size}) as bin"
).filter(f"bin < {num_bins}")
bin_counts = data_with_bin.groupBy("bin").count()
    .orderBy("bin")
bin_counts_list = bin_counts.collect()
bin_indices = []
bin_values = []
for row in bin_counts_list:
    bin_indices.append(row['bin'])
    bin_values.append(row['count'])
bin_centers = [min_value + (bin_idx + 0.5) * bin_size 
    for bin_idx in bin_indices]
\end{code}

Полученные результаты показывают, что cтоимость продуктов пользователей смещена в сторону невысоких значений (мода 125–150)  (см. рис. \ref{fig:PriceField}).

\begin{figure}[H]
    \centering
    \includegraphics[width=1.02\textwidth]{Content/Images/Analyze/PriceField.png}
    \caption{Гистограмма распределения для price}
    \label{fig:PriceField}
\end{figure}

На рисунках \ref{fig:PriceAnomaly1}, \ref{fig:PriceAnomaly2} продемонстрировано распределение аномальных значений у price. Фукнция построения и расчетов данных для ящика с усами вынесена в Приложение \ref{app:plot_boxplots}.

Вычисление квантилей и медианы для построения ящика с усами выполняется в следующем фрагменте:

\begin{code}
for column in columns:
    quantiles = data.approxQuantile(column, 
        [0.25, 0.5, 0.75], 0.01)
    q1, median, q3 = quantiles
\end{code}

Вычисление границ и межквартильного размаха для построения ящика с усами выполняется в следующем фрагменте:

\begin{code}
iqr = q3 - q1
lower_bound = q1 - 1.5 * iqr
upper_bound = q3 + 1.5 * iqr
\end{code}

Ограничение усов минимальным и максимальным значениями в графике выполняется в следующем фрагменте:

\begin{code}
lower_bound = max(lower_bound, min_value)
upper_bound = min(upper_bound, max_value)
\end{code}

Вычисление среднеквадратичного отклоненения, среднего, минимального и максимального значений для вывода статистических характеристик:

\begin{code}
min_value = data.agg({column: "min"}).collect()[0][0]
mean_value = data.agg({column: "mean"}).collect()[0][0]
std_value = data.agg({column: "std"}).collect()[0][0]
max_value = data.agg({column: "max"}).collect()[0][0]
print(f"Минимальное значение:          {min_value:.2f}")
print(f"Среднее значение:              {mean_value:.2f}")
print(f"Среднеквадратичное отклонение: {std_value:.2f}")
print(f"Первый квартиль:               {q1:.2f}")
print(f"Медиана:                       {median:.2f}")
print(f"Третий квартиль:               {q3:.2f}")
print(f"Максимальное значение:         {max_value:.2f}")
\end{code}

\begin{figure}[H]
    \centering
    \includegraphics[width=1\textwidth]{Content/Images/Analyze/PriceAnomaly1.png}
    \caption{Пример аномалий у price}
    \label{fig:PriceAnomaly1}
\end{figure}

\begin{figure}[H]
    \centering
    \includegraphics[width=0.6\textwidth]{Content/Images/Analyze/PriceAnomaly2.png}
    \caption{Расчетные значения у price}
    \label{fig:PriceAnomaly2}
\end{figure}

Для категориальных признаков \texttt{category\_code}, \texttt{brand} были проанализированы частоты встречаемости. Самым популярным значением для поля:
\begin{itemize}
\item \texttt{brand} стало samsung (рис. \ref{fig:BrandCategoriesHist});
\item \texttt{category\_code} стало electronics (рис. \ref{fig:CategoryCodeHist}).
\end{itemize} 

Расчет данных для построения графиков выполняется в следующем фрагменте кода:
\begin{code}
column_type = dict(data.dtypes)[column_name]
if column_type == 'array<string>':
    categories = (data
        .select(explode(col(column_name)).alias(column_name))
        .groupBy(column_name).count().orderBy("count", 
            ascending=False)
    )
else:
    categories = (data
        .groupBy(column_name).count().orderBy("count",
             ascending=False)
    )
total_categories = categories.count()
categories_list = categories.limit(top_n).collect()
category_names = []
category_counts = []
for row in categories_list:
    category_names.append(row[column_name])
    category_counts.append(row['count'])
\end{code}

\begin{figure}[H]
    \centering
    \includegraphics[width=0.85\textwidth]{Content/Images/Analyze/BrandCategoriesHist.png}
    \caption{Частоты для brand}
    \label{fig:BrandCategoriesHist}
\end{figure}

\begin{figure}[H]
    \centering
    \includegraphics[width=0.85\textwidth]{Content/Images/Analyze/CategoryCodeHist.png}
    \caption{Частоты для category\_code}
    \label{fig:CategoryCodeHist}
\end{figure}

Проверка дубликатов показала, что некоторые товары имеют повторяющиеся записи — это связано с тем, что каждая запись соответствует отдельному пользовательскому действию. Такие дубликаты являются ожидаемыми и отражают структуру исходного набора данных. Поэтому для всего датасета была проведена дедубликация по \texttt{product\_id}, \texttt{event\_type} и \texttt{price} полям командой \texttt{df.dropDuplicates(["product\_id", "event\_type", "price"])}.

\vspace{\baselineskip}

\section{Выводы}\vspace{\baselineskip}

В ходе проведённого разведочного анализа был сформирован целостный и очищенный набор данных, готовый для применения алгоритмов машинного обучения. Основными результатами являются:

\begin{itemize}
\item выполнена загрузка данных из HDFS средствами Apache Spark;
\item исследована структура данных, выявлены и обработаны пропуски и аномалии;
\item проведён анализ распределений и выбросов в количественных характеристиках;
\item проведён анализ частот встречаемости значений в категориальных характеристиках;
\item сформированы новые признаки, повышающие информативность дан
ных;
\item выявлены особенности набора данных, связанные с дублированием записей по идентификатору продукта.
\end{itemize}
% Содержание второй главы
%
\chapter{\MakeUppercase{Машинное обучение на больших данных}}\label{ch:second}

\vspace{\baselineskip}

В данной главе рассматриваются методы построения и оценки моделей машинного обучения в распределённой среде Apache Spark \cite{karau2015spark, damji2020learning, chambers2018spark, armbrust2015spark}. Работа включает решение двух задач: прогнозирования числовой оценки цены товара (регрессия) и классификации бюджетного класса продукта. Все вычисления выполнялись с использованием фреймворка Apache Spark и библиотеки Spark ML \cite{Tekdogan2022, kaggle2023dataset}, обеспечивающих обработку данных объёмом около трёх миллионов записей.

\vspace{\baselineskip}

\section{Задача регрессии}
\subsection{Постановка задачи регрессии}\vspace{\baselineskip}

Необходимо построить модель GBT регрессии для предсказания цены продукта (price) на основе доступных признаков.
Цель: найти нелинейную зависимость между признаками и целевой переменной, минимизируя ошибку предсказания. Требуется использовать RMSE и R² для оценки качества обучения модели. Модель должна объяснить, какие факторы влияют на цену товара и насколько сильно.

В качестве признаков были выделены следующие группы:

\begin{code}
binary_features = [
    "is_expensive", 
    "is_budget", 
    "is_mid_range", 
    "is_purchase", 
    "is_view", 
    "is_cart",
    "contains_appliances",
    "contains_computers",
    "contains_electronics",
    "contains_kitchen",
    "contains_smartphone"
]

numeric_features = ["category_count"]

categorical_features = ["brand", "event_type", 
                        "price_range", "price_range_numeric"]
\end{code}

проводилось обучение модели.

Из анализа были исключены признаки, не влияющие на целевую переменную или потенциально приводящие к переобучению: идентификаторы (\texttt{category\_id}, \texttt{event\_time}, \texttt{user\_id}, \texttt{user\_session}) (см. рис. \ref{fig:ExampleData2}-\ref{fig:SchemeAfterProcessing2}).

\begin{figure}[H]
    \centering
    \includegraphics[width=1\textwidth]{Content/Images/Analyze/ExampleData.png}
    \caption{Фрагмент датафрейма с исходными данными}
    \label{fig:ExampleData2}
\end{figure}

\begin{figure}[H]
    \centering
    \includegraphics[width=1\textwidth]{Content/Images/Analyze/ExampleDataAfterProcessing.png}
    \caption{Фрагмент датафрейма с обработанными данными}
    \label{fig:ExampleDataAfterProcessing2}
\end{figure}

\begin{figure}[H]
    \centering
    \includegraphics[width=0.8\textwidth]{Content/Images/Analyze/SchemeAfterProcessing.png}
    \caption{Схема данных}
    \label{fig:SchemeAfterProcessing2}
\end{figure}

Для оценки качества модели использовались метрики RMSE (Root Mean Square Error) и R² (коэффициент детерминации).

\vspace{\baselineskip}\subsection{Решение задачи регрессии}\vspace{\baselineskip}

Построение модели линейной регрессии начиналось с подготовки данных: датасет загружался из HDFS в формате Parquet, после чего из него выделялся небольшой сэмпл для последующей потоковой обработки. Основная выборка разделялась на тренировочную и тестовую части в соотношении 80/20 (см. рис. \ref{fig:DatasetSize}).

\begin{code}
spark.read.parquet("hdfs://namenode:9000/user/dchel/dchel_datab
ase/eCommerce_clear_data")
\end{code}

\vspace{\baselineskip}

Далее выполнялась предобработка признаков. Категориальные параметры преобразовывались с помощью StringIndexer. Все признаки объединялись в единый вектор посредством VectorAssembler.

На основе этих этапов формировался конвейер Spark ML, включающий индексацию, кодирование, масштабирование признаков и модель GBT регрессии. Для неё использовались параметры maxIter=100, maxDepth=5, regParam=0.01.

Оптимизация модели выполнялась с помощью 2-кратной кросс-валидации. Наилучшие результаты показала конфигурация с regParam=0.1, maxIter=100 и maxDepth=5 (см. рис. \ref{fig:BestGBT}).

\begin{code}
CrossValidator(estimator=pipeline,
                estimatorParamMaps=param_grid,
                evaluator=cv_evaluator,
                numFolds=2,
                parallelism=4)
\end{code}

\begin{figure}[H]
    \centering
    \includegraphics[width=0.7\textwidth]{Content/Images/ML/BestGBT.png}
    \caption{Конфигурация лучшей GBT модели}
    \label{fig:BestGBT}
\end{figure}

\vspace{\baselineskip}\subsection{Анализ полученных результатов регрессии}\vspace{\baselineskip}

Модель была протестирована на отложенной тестовой выборке. Получены следующие значения метрик (см. рис. \ref{fig:MetricsGBT}):

\begin{itemize}
\item RMSE = 65.9345;
\item R² = 0.7862.
\end{itemize}

\begin{figure}[H]
    \centering
    \includegraphics[width=0.7\textwidth]{Content/Images/ML/MetricsGBT.png}
    \caption{RMSE и R² метрики}
    \label{fig:MetricsGBT}
\end{figure}

Высокое значение R² свидетельствует о сильной объясняющей способности модели.

\vspace{\baselineskip}

\section{Задача классификации с использованием LogisticRegression}
\subsection{Постановка задачи классификации}\vspace{\baselineskip}

Вторая часть работы посвящена построению модели многоклассовой классификации, определяющей бюджетный класс (price\_range\_numeric). Целевой переменной является числовой индикатор бюджетной категории.

Требуется построить классификатор на основе Logistic Regression, предсказывающий бюджетный класс по доступным данным. Необходимо проанализировать работу модели на валидационной выборке, определить оптимальный порог принятия решения и представить модель, которая, гарантируя обнаружение не менее 60\% всех полезных отзывов (Recall ≥ 0.60), обеспечивает при этом наивысшую возможную долю верных предсказаний среди всех отмеченных как полезные (Precision).

\vspace{\baselineskip}\subsection{Решение задачи классификации}\vspace{\baselineskip}

В задаче классификации данные также загружались из HDFS. Потом производилось разделение тестовую и обучающую выборки (\ref{fig:DatasetSize}).

\begin{figure}[H]
    \centering
    \includegraphics[width=0.6\textwidth]{Content/Images/ML/DatasetSize.png}
    \caption{Объём выборок и кол-во экземпляров классов}
    \label{fig:DatasetSize}
\end{figure}

На этапе предобработки категориальные признаки преобразовывались с помощью StringIndexer, после чего все признаки объединялись в общий вектор, необходимый для обучения модели. Нормализация числовых признаков проводилась с помощью StandardScaler.

Процесс обучения реализовывался через конвейер, включающий этапы подготовки данных и модель LogisticRegression. Для начального варианта использовались параметры maxIter=100, regParam=0.01, elasticNetParam=0.0 и family="multinomial".

Оптимизация гиперпараметров выполнялась с использованием 2-кратной кросс-валидации. Наилучшие результаты были достигнуты при maxIter=100, regParam=0.01, elasticNetParam=1.0 (см. рис. \ref{fig:BestLR}):

\begin{code}
param_grid = ParamGridBuilder() \
    .addGrid(lr_model.regParam, [0.01, 0.1, 1.0]) \
    .addGrid(lr_model.elasticNetParam, [0.0, 0.5, 1.0]) \
    .addGrid(lr_model.maxIter, [10, 100]) \
    .build()
CrossValidator(estimator=pipeline,
                estimatorParamMaps=param_grid,
                evaluator=cv_evaluator,
                numFolds=2,
                parallelism=4)
\end{code}

\begin{figure}[H]
    \centering
    \includegraphics[width=0.7\textwidth]{Content/Images/ML/BestLR.png}
    \caption{Конфигурация лучшей LR модели}
    \label{fig:BestLR}
\end{figure}

\vspace{\baselineskip}
\subsection{Анализ полученных результатов классификации}
\vspace{\baselineskip}

Модель классификации продемонстрировала стабильные результаты: точность (Precision) составила 0.97, полнота (Recall) — 0.968, F1-мера — 0.967, а общая точность классификации достигла 0.977 (см. рис. \ref{fig:MetricsLR}).

\begin{figure}[H]
    \centering
    \includegraphics[width=\textwidth]{Content/Images/ML/MetricsLR.png}
    \caption{Метрики LR модели}
    \label{fig:MetricsLR}
\end{figure}

Матрица ошибок для выбранного порога показывает следующие значения: более 12 000 объектов были корректно классифицированы, около 2 000 — некорректно (см. рис. \ref{fig:ConfusionMatrix} и Приложение \ref{app:confusion_matrix}).

\begin{figure}[H]
    \centering
    \includegraphics[width=0.8\textwidth]{Content/Images/ML/ConfusionMatrix.png}
    \caption{Матрица ошибок}
    \label{fig:ConfusionMatrix}
\end{figure}

График распределения кол-ва верно предсказанных классификаций показывает следующее (см. Приложение \ref{app:hist_raspred}):

\begin{figure}[H]
    \centering
    \includegraphics[width=0.8\textwidth]{Content/Images/ML/PredictHist.png}
    \caption{Кол-во верно предсказанных классификаций}
    \label{fig:PredictHist}
\end{figure}

\vspace{\baselineskip}\section{Выводы}\vspace{\baselineskip}

В рамках работы были решены две задачи машинного обучения. GBT регрессия показала эффективность: значение R² (0.78) подтверждает, что признаков достаточно для точного прогнозирования цены продуктов. Задача классификации оказалась более успешной: модель достигла accurancy = 0.96, F1 = 0.96 и выполнила требуемый уровень полноты (Recall ≥ 60\%).

В перспективе дальнейшее развитие связано с использованием текстовых признаков \cite{zaharia2021lakehouse} (TF-IDF, Word2Vec, BERT), внедрением нейронных сетей и современных ансамблей, а также сокращением размерности категориальных признаков. Полученные результаты демонстрируют эффективность Spark ML при анализе продуктов электронной коммерции в условиях больших данных.

% ========== Заключение ==========
\chapter*{Заключение}
\addcontentsline{toc}{chapter}{\MakeUppercase{Заключение}}
% Здесь пишется заключение по работе
%

\vspace{\baselineskip}
В ходе выполнения курсовой работы был реализован полный цикл анализа больших данных и машинного обучения на примере датасета <<eCommerce behavior data from multi category store>> с использованием фреймворка Apache Spark. В рамках исследования проведён разведочный анализ данных, включающий загрузку, очистку и преобразование информации источника. Были обработаны пропущенные значения, созданы новые признаки, проанализированы распределения и выбросы, что обеспечило подготовку качественного набора данных для последующего моделирования.

Для решения поставленных задач были реализованы и протестированы две модели машинного обучения. GBT регрессия для прогнозирования цены товара показала устойчивую сходимость, высокое значение $R^2 = 0,78$. Модель логистической регрессии для классификации бюджетного класса продуктов продемонстрировала хорошее качество (accurancy = 0,967) и выполнила поставленное условие: полнота (Recall) не ниже 60\% при точности (Precision) 97\% и при следующих значениях гиперпараметров: maxIter=100, regParam=0.01, elasticNetParam=1.0.

Сравнительный анализ задач показал, что бинарные признаки, такие как is\_expensive или is\_budget, эффективнее используются для прогнозирования цены на продукт. Применение распределённых вычислений на платформе Apache Spark подтвердило свою эффективность при работе с большими объёмами данных, обеспечив масштабируемость и высокую производительность на всех этапах проекта.

Перспективы дальнейшего развития работы включают:

\begin{itemize}
\item расширение анализируемых пользовательских действий для построения полного воронки конверсии и расчета CTR (Click-Through Rate);
\item обогащение данных профилями пользователей с применением RFM-анализа (Recency, Frequency, Monetary) и сегментацией по предпочтениям;
\item внедрение временных и сезонных признаков для учета суточных, недельных и праздничных паттернов активности;
\item разработку рекомендательных систем на основе контентной фильтрации (content-based) и коллаборативной фильтрации (collaborative filtering);
\item анализ путей пользователей (Customer Journey) с применением марковских цепей для моделирования переходов между категориями;
\item реализацию динамического ценообразования на основе анализа эластичности спроса и конкурентной среды;
\item восстановление и структурирование иерархии категорий товаров для кросс-категорийного анализа;
\item прогнозирование спроса с использованием моделей временных рядов (ARIMA/SARIMA, Prophet) и нейронных сетей (LSTM).
\end{itemize}

% ========== Список литературы ==========
\clearpage
\printbibliography[heading=bibintoc]

% ========== Приложения ==========
\appendix
\renewcommand*{\printchaptername}{\MakeUppercase{\appendixname}}
\renewcommand*{\afterchapternum}{\centering\par\nobreak\vskip \midchapskip}
\renewcommand*{\printchapternum}{\chapnumfont \thechapter}
\renewcommand\thechapter{\Asbuk{chapter}}
\renewcommand*{\printchaptertitle}[1]{\chaptitlefont #1}

% Здесь пишется содержание приложений
%
\chapter{Обработка данных}
\label{app:transform_data}\vspace{\baselineskip}

\begin{code}
def transform_dataframe(data: DataFrame) -> DataFrame:
    df = df.select(
        "event_type", "product_id", "category_id", "category_code", "brand", "price"
    )
    data = data.withColumn("category_id", col("category_id").cast("Integer"))
    data = data.withColumn("product_id", col("product_id").cast("Integer"))
    data = data.withColumn("price", col("price").cast("Float"))
    # Преобразуем строку в массив строк
    data = data.withColumn("category_code", 
                            split(col("category_code"), r"\.")
                          )
    data = data.dropna(subset=["category_code", "brand"])
    df = df.withColumn("is_expensive", when(col("price") > 200, 1).otherwise(0))
    df = df.withColumn("is_budget", when(col("price") < 50, 1).otherwise(0))
    df = df.withColumn("is_mid_range", when((col("price") >= 50) & (col("price") <= 200), 1).otherwise(0))
    df = df.withColumn("category_count", 
                   col("contains_appliances").cast("int") + 
                   col("contains_computers").cast("int") + 
                   col("contains_electronics").cast("int") + 
                   col("contains_kitchen").cast("int") + 
                   col("contains_smartphone").cast("int"))
    df = df.withColumn("is_purchase", when(col("event_type") == "purchase", 1).otherwise(0))
    df = df.withColumn("is_view", when(col("event_type") == "view", 1).otherwise(0))
    df = df.withColumn("is_cart", when(col("event_type") == "cart", 1).otherwise(0))
    df = df.withColumn("price_range", 
                   when(col("price") < 50, "budget")
                   .when(col("price") < 150, "affordable")
                   .when(col("price") < 300, "premium")
                   .otherwise("luxury"))

    # Создаем числовое представление для price_range
    df = df.withColumn("price_range_numeric",
                    when(col("price_range") == "budget", 1)
                    .when(col("price_range") == "affordable", 2)
                    .when(col("price_range") == "premium", 3)
                    .otherwise(4))
    return df
\end{code}

\end{document}
