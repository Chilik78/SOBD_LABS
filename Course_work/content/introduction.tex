% Содержание введения
%
\vspace{\baselineskip}

Актуальность данной курсовой работы обусловлена возрастающей потребностью в разработке эффективных методов машинногообучениядля обработки больших данных \cite{karau2015spark, damji2020learning, chambers2018spark} и извлечения полезных знаний из пользовательского контента \cite{koirala2020pyspark}. Особую значимость приобретают комплексные подходы, сочетающие предварительную обработку данных и построение прогностических моделей на распределенных вычислительных платформах.

В связи с этим целью данной курсовой работы является исследование и реализация полного цикла машинного обучения на больших данных о поведении пользователя на рынке электронной коммерции с использованием фреймворка Apache Spark \cite{spark2022official, armbrust2015spark}.

Для достижения поставленной цели выдвинуты следующие задачи:
\begin{enumerate}
    \item Загрузка и первичное исследование структуры данных из распределенной файловой системы HDFS;
    \item Выполнение базовых преобразований и очистки данных для подготовки к машинному обучению;
    \item Применение алгоритмов машинного обучения на больших данных;
    \item Анализ и оценка качества построенных прогностических моделей;
    \item Визуализация результатов и подготовка выводов по исследованию.
\end{enumerate}

Впервом разделе рассмотрена более подробно постановка задачи и проведен обзор современных методов машинного обучения на больших данных. Во втором разделе описана методика предварительной обработки данных и выполнена верификация качества очистки. В третьем разделе представлены результаты применения алгоритмов машинного обучения, а также анализ эффективности построенных моделей. В заключении работы сформулированы общие выводы по результатам проведенного исследования.