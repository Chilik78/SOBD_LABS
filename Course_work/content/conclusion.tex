% Здесь пишется заключение по работе
%

\vspace{\baselineskip}
В ходе выполнения курсовой работы был реализован полный цикл анализа больших данных и машинного обучения на примере датасета <<eCommerce behavior data from multi category store>> с использованием фреймворка Apache Spark. В рамках исследования проведён разведочный анализ данных, включающий загрузку, очистку и преобразование информации источника. Были обработаны пропущенные значения, созданы новые признаки, проанализированы распределения и выбросы, что обеспечило подготовку качественного набора данных для последующего моделирования.

Для решения поставленных задач были реализованы и протестированы две модели машинного обучения. GBT регрессия для прогнозирования цены товара показала устойчивую сходимость, высокое значение $R^2 = 0,78$. Модель логистической регрессии для классификации бюджетного класса продуктов продемонстрировала хорошее качество (accurancy = 0,967) и выполнила поставленное условие: полнота (Recall) не ниже 60\% при точности (Precision) 97\% и при следующих значениях гиперпараметров: maxIter=100, regParam=0.01, elasticNetParam=1.0.

Сравнительный анализ задач показал, что бинарные признаки, такие как is\_expensive или is\_budget, эффективнее используются для прогнозирования цены на продукт. Применение распределённых вычислений на платформе Apache Spark подтвердило свою эффективность при работе с большими объёмами данных, обеспечив масштабируемость и высокую производительность на всех этапах проекта.

Перспективы дальнейшего развития работы включают:

\begin{itemize}
\item расширение анализируемых пользовательских действий для построения полного воронки конверсии и расчета CTR (Click-Through Rate);
\item обогащение данных профилями пользователей с применением RFM-анализа (Recency, Frequency, Monetary) и сегментацией по предпочтениям;
\item внедрение временных и сезонных признаков для учета суточных, недельных и праздничных паттернов активности;
\item разработку рекомендательных систем на основе контентной фильтрации (content-based) и коллаборативной фильтрации (collaborative filtering);
\item анализ путей пользователей (Customer Journey) с применением марковских цепей для моделирования переходов между категориями;
\item реализацию динамического ценообразования на основе анализа эластичности спроса и конкурентной среды;
\item восстановление и структурирование иерархии категорий товаров для кросс-категорийного анализа;
\item прогнозирование спроса с использованием моделей временных рядов (ARIMA/SARIMA, Prophet) и нейронных сетей (LSTM).
\end{itemize}